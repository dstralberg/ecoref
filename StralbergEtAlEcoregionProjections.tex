\documentclass[]{article}
\usepackage{lmodern}
\usepackage{amssymb,amsmath}
\usepackage{ifxetex,ifluatex}
\usepackage{fixltx2e} % provides \textsubscript
\ifnum 0\ifxetex 1\fi\ifluatex 1\fi=0 % if pdftex
  \usepackage[T1]{fontenc}
  \usepackage[utf8]{inputenc}
\else % if luatex or xelatex
  \ifxetex
    \usepackage{mathspec}
  \else
    \usepackage{fontspec}
  \fi
  \defaultfontfeatures{Ligatures=TeX,Scale=MatchLowercase}
\fi
% use upquote if available, for straight quotes in verbatim environments
\IfFileExists{upquote.sty}{\usepackage{upquote}}{}
% use microtype if available
\IfFileExists{microtype.sty}{%
\usepackage{microtype}
\UseMicrotypeSet[protrusion]{basicmath} % disable protrusion for tt fonts
}{}
\usepackage[margin=1in]{geometry}
\usepackage{hyperref}
\hypersetup{unicode=true,
            pdftitle={Climate-projected distributional shifts and refugia for North American ecoregions},
            pdfauthor={Diana Stralberg (diana.stralberg@ualberta.ca), AdaptWest Project},
            pdfborder={0 0 0},
            breaklinks=true}
\urlstyle{same}  % don't use monospace font for urls
\usepackage{longtable,booktabs}
\usepackage{graphicx,grffile}
\makeatletter
\def\maxwidth{\ifdim\Gin@nat@width>\linewidth\linewidth\else\Gin@nat@width\fi}
\def\maxheight{\ifdim\Gin@nat@height>\textheight\textheight\else\Gin@nat@height\fi}
\makeatother
% Scale images if necessary, so that they will not overflow the page
% margins by default, and it is still possible to overwrite the defaults
% using explicit options in \includegraphics[width, height, ...]{}
\setkeys{Gin}{width=\maxwidth,height=\maxheight,keepaspectratio}
\IfFileExists{parskip.sty}{%
\usepackage{parskip}
}{% else
\setlength{\parindent}{0pt}
\setlength{\parskip}{6pt plus 2pt minus 1pt}
}
\setlength{\emergencystretch}{3em}  % prevent overfull lines
\providecommand{\tightlist}{%
  \setlength{\itemsep}{0pt}\setlength{\parskip}{0pt}}
\setcounter{secnumdepth}{0}
% Redefines (sub)paragraphs to behave more like sections
\ifx\paragraph\undefined\else
\let\oldparagraph\paragraph
\renewcommand{\paragraph}[1]{\oldparagraph{#1}\mbox{}}
\fi
\ifx\subparagraph\undefined\else
\let\oldsubparagraph\subparagraph
\renewcommand{\subparagraph}[1]{\oldsubparagraph{#1}\mbox{}}
\fi

%%% Use protect on footnotes to avoid problems with footnotes in titles
\let\rmarkdownfootnote\footnote%
\def\footnote{\protect\rmarkdownfootnote}

%%% Change title format to be more compact
\usepackage{titling}

% Create subtitle command for use in maketitle
\newcommand{\subtitle}[1]{
  \posttitle{
    \begin{center}\large#1\end{center}
    }
}

\setlength{\droptitle}{-2em}
  \title{Climate-projected distributional shifts and refugia for North American
ecoregions}
  \pretitle{\vspace{\droptitle}\centering\huge}
  \posttitle{\par}
  \author{Diana Stralberg
(\href{mailto:diana.stralberg@ualberta.ca}{\nolinkurl{diana.stralberg@ualberta.ca}}),
AdaptWest Project}
  \preauthor{\centering\large\emph}
  \postauthor{\par}
  \date{}
  \predate{}\postdate{}


\begin{document}
\maketitle

Climate model projections suggest major North American biome shifts in
response to anthropogenic climate change (Rehfeldt et al. 2012). Such
shifts could have profound influences on native flora and fauna, many of
which would have to move long distances to track their climatic niches.
To evaluate potential ecosystem changes at a somewhat finer scale, I
projected the change in climate space for level III ecoregions
(Commission for Environmental Cooperation 1997) as surrogates for
multiple associated species and ecological communities. First, I
developed a random forest model (Breiman 2001) to predict ecoregion
class from bioclimatic variables. I used 10-km interpolated climate data
for the 1971-2000 normal period, available from Natural Resources Canada
(McKenney et al. 2011)

R Code for this portion follows:

\begin{verbatim}
library(randomForest)
library(raster)

#eco = project directory
#datLL = data frame of lat-lon sample points for which to extract climate variables ("CECEcoregionSampleLL.csv"")
#cececo = data frame with ecoregion names ("CECecoregions.csv")
#ecolevel3r = ecoregion raster with a Lambert azimuthal equal-area projection ("ceclev3idlaz.tif")
#ecolevel3s = ecoregion shapefile with a Lambert azimuthal equal-area projection ("NA_CEC_Eco_Level3_lazea.shp")
lazea <-  CRS("+proj=laea +lat_0=45 +lon_0=-100 +x_0=0 +y_0=0 +datum=WGS84 +units=m +no_defs +ellps=WGS84 +towgs84=0,0,0")

#cur = directory containing grids representing derived climate variables
setwd(cur)
clim <- list.files(cur, pattern =".asc$")
curclim<-stack(clim)

temp <- raster(clim[1])
ID <- as.data.frame(rasterToPoints(temp))
names(ID)[3] <- "ID4km"
ID$ID <- row.names(ID)
IDR <- raster(ncols=ncol(temp), nrows=nrow(temp), xmn=xmin(temp), xmx=xmax(temp), ymn=ymin(temp), ymx=ymax(temp))
IDRR <- rasterize(as.matrix(ID[,1:2]), IDR, as.numeric(ID[,4]))

curclim <- addLayer(curclim,IDRR)

sampleclim<-cbind(datLL,extract(curclim,as.matrix(cbind(datLL[,3],datLL[,4]))))
sc <- na.omit(sampleclim)
names(sc)[ncol(sc)] <- "IDgrid"
sc$NA_L3CODE <- as.factor(as.character(sc$NA_L3CODE))
lu <- as.data.frame(levels(sc$NA_L3CODE))
lu$level <- row.names(lu)
names(lu)[1] <- "NA_L3CODE"

setwd(eco)
eco.rf <- randomForest(y=sc$NA_L3CODE, x=sc[,5:(ncol(sc)-1)],importance = TRUE, proximity = TRUE, data=sc)
eco1.rf <- randomForest(y=sc$NA_L3CODE, x=sc[,5:23],importance = TRUE, proximity = TRUE, data=sc)
round(importance(eco1.rf), 2)
varImpPlot(eco1.rf) 
ecocurr <- predict(curclim,eco1.rf)
ecocurrlaz <- round(projectRaster(ecocurr, ecolevel3r, method='ngb'),0)
writeRaster(ecocurrlaz,filename="current1_lazea.tif",datatype='INT4S',format="GTiff",overwrite=TRUE)
curfreq <- freq(ecocurrlaz)
ecolu <- merge(lu,curfreq,by.x="level",by.y="value")
names(ecolu)[3] <- "curr"
\end{verbatim}

This model was then used to project ecoregions onto future mid-century
(2041-2070) and end-of-century (2071-2100) climate conditions. Climate
projections were based on 10-km downscaled climate anomalies (McKenney
et al. 2011) generated by four widely-used GCMs from the Coupled Model
Intercomparison Project, Phase 5 (CMIP5, Taylor et al. 2012): CanESM2,
CESM1-CAM5, HadGEM2-ES, and MIROC-ESM. These particular GCMs were
selected for downscaling by the Canadian Forest Service based on
availability of key variables such as solar radiation, wind speed and
humidity, as well as temperature and precipitation to support various
forest modeling efforts (McKenney et al., 2013). We used representative
concentration pathway (RCP) 8.5, to represent the 21st century
conditions that are to be expected without dramatic reductions in
greenhouse gas emissions or technological fixes (Fuss et al. 2014).

The following code generates projections for each GCM and time period
and reprojects the results to an equal-area projection.

\begin{verbatim}
#fut = directory containing grids representing derived future climate variables
gcm <- c("canesm2","cesm1cam5","hadgem2es","mirocesm")
rcp <- c("rcp45","rcp85")
time <- c("2041-2070","2071-2100")
for (i in gcm) {
    for (j in rcp) {
        for (k in time) {
            w  <- paste(fut,i,"/",j,"/",k,"/",sep="")
            setwd(w)
            futclim <- list.files(w,pattern=".asc$")
            s <-stack(futclim[1:99])
            p <- predict(s,eco1.rf)
            plaz <- round(projectRaster(p,ecolevel3r,method='ngb'),0)
            writeRaster(plaz, filename=paste(eco,j,k,i,"1",sep="_"),datatype='INT4S',format="GTiff", overwrite=TRUE)
            }
        }   
    }
\end{verbatim}

Next, I identified the most frequently-predicted Level-III ecoregion
(Table 1) at each pixel location (i.e, the mode) for RCP 4.5 and RCP
8.5.

\begin{verbatim}
groups <- c("rcp45_2041-2070","rcp45_2071-2100","rcp85_2041-2070","rcp85_2071-2100")
setwd(eco)
for (i in groups) {
    g <- list.files(eco,pattern=i)
    g1 <- grep(pattern=".tif$",g,value=TRUE)
    gs <- stack(g1)
    m <- overlay(gs, fun = modal)   
    futfreq <- as.data.frame(freq(m))
    names(futfreq)[2] <- i
    ecolu <- merge(ecolu,futfreq,by.x="level",by.y="value")
    writeRaster(m, filename=paste(eco,i,"mode",sep="_"),datatype='INT4S',format="GTiff", overwrite=TRUE)
    }
ecolu1 <- merge(unique(cececo[,c(2:4)]),ecolu[,2:7],by="NA_L3CODE")
write.csv(ecolu1,file="ecoregion_change.summary.csv",row.names=FALSE)
\end{verbatim}

\begin{figure}
\centering
\includegraphics{I:/GoogleDrive/COBB_ClimateChange/ACE-ECOResubmission/ecoref/ecoregion_cmip5_RCP45.png}
\caption{Figure 1. Model-predicted (a) baseline, (b) mid-century, and
(c) end-of-century changes in North American ecoregions for RCP 4.5.
Boreal and western forested regions are shown in green and blue-green
shades; arctic ecoregions are in purple shades; prairie/parkland
ecoregions are in red-brown shades; and temperate forest ecoregions are
in light green, yellow, and orange shades (see Table 1 for full list of
ecoregions). Boreal ecoregions are also outlined in black.}
\end{figure}

\begin{figure}
\centering
\includegraphics{I:/GoogleDrive/COBB_ClimateChange/ACE-ECOResubmission/ecoref/ecoregion_cmip5_RCP85.png}
\caption{Figure 2. Model-predicted (a) baseline, (b) mid-century, and
(c) end-of-century changes in North American ecoregions for RCP 8.5.
Boreal and western forested regions are shown in green and blue-green
shades; arctic ecoregions are in purple shades; prairie/parkland
ecoregions are in red-brown shades; and temperate forest ecoregions are
in light green, yellow, and orange shades (see Table 1 for full list of
ecoregions). Boreal ecoregions are also outlined in black.}
\end{figure}

I then calculated the change in area (16 km2 pixels) for each Level III
ecoregion:

Table 1. Model-projected changes by ecoregion:

\begin{longtable}[]{@{}llrrrrr@{}}
\toprule
NA\_L3CODE & NA\_L3NAME & curr & rcp45\_2041.2070 & rcp45\_2071.2100 &
rcp85\_2041.2070 & rcp85\_2071.2100\tabularnewline
\midrule
\endhead
1.1.2 & Baffin and Torngat Mountains & 5785 & 5197 & 2250 & 1965 &
393\tabularnewline
10.1.1 & Thompson-Okanogan Plateau & 3839 & 5693 & 3589 & 3376 &
2862\tabularnewline
10.1.2 & Columbia Plateau & 5036 & 5531 & 4827 & 5101 &
5167\tabularnewline
10.1.3 & Northern Basin and Range & 9141 & 1445 & 334 & 647 &
11\tabularnewline
10.1.4 & Wyoming Basin & 8515 & 1245 & 560 & 448 & 363\tabularnewline
10.1.5 & Central Basin and Range & 16892 & 15798 & 14737 & 13595 &
5904\tabularnewline
10.1.6 & Colorado Plateaus & 8002 & 14825 & 14476 & 14470 &
15813\tabularnewline
10.1.7 & Arizona/New Mexico Plateau & 8877 & 12827 & 11906 & 10181 &
5301\tabularnewline
10.1.8 & Snake River Plain & 4042 & 5396 & 6074 & 5801 &
1949\tabularnewline
10.2.1 & Mojave Basin and Range & 7766 & 10443 & 13446 & 16203 &
32738\tabularnewline
10.2.2 & Sonoran Desert & 7208 & 13908 & 16462 & 17540 &
27624\tabularnewline
10.2.4 & Chihuahuan Desert & 10202 & 15753 & 14807 & 14405 &
12389\tabularnewline
11.1.1 & California Coastal Sage, Chaparral, and Oak Woodlands & 5946 &
8910 & 10364 & 10810 & 19407\tabularnewline
11.1.2 & Central California Valley & 3208 & 2566 & 2364 & 2829 &
1070\tabularnewline
11.1.3 & Southern and Baja California Pine-Oak Mountains & 1794 & 1249 &
1059 & 998 & 1014\tabularnewline
12.1.1 & Madrean Archipelago & 3214 & 4902 & 4809 & 5281 &
6232\tabularnewline
13.1.1 & Arizona/New Mexico Mountains & 7069 & 5884 & 6871 & 6695 &
7784\tabularnewline
15.4.1 & Southern Florida Coastal Plain & 2542 & 4479 & 5451 & 5394 &
7981\tabularnewline
2.1.4 & Lancaster and Borden Peninsula Plateaus & 5018 & 321 & 118 & 152
& 9\tabularnewline
2.1.5 & Foxe Uplands & 25575 & 4649 & 4173 & 3141 & 1985\tabularnewline
2.1.6 & Baffin Uplands & 7702 & 1641 & 919 & 1122 & 984\tabularnewline
2.1.7 & Gulf of Boothia and Foxe Basin Plains & 11438 & 400 & 128 & 51 &
75\tabularnewline
2.1.9 & Banks Island and Amundsen Gulf Lowlands & 11202 & 16289 & 9323 &
8539 & 68\tabularnewline
2.2.1 & Arctic Coastal Plain & 4145 & 12223 & 13419 & 13942 &
2073\tabularnewline
2.2.2 & Arctic Foothills & 8286 & 37118 & 49520 & 57446 &
46094\tabularnewline
2.2.3 & Subarctic Coastal Plains & 6951 & 38956 & 51348 & 55642 &
96698\tabularnewline
2.2.4 & Seward Peninsula & 3685 & 14046 & 12792 & 12515 &
4386\tabularnewline
2.2.5 & Bristol Bay-Nushagak Lowlands & 4289 & 15780 & 18318 & 17968 &
23889\tabularnewline
2.3.1 & Brooks Range/Richardson Mountains & 9438 & 4093 & 2236 & 2351 &
1008\tabularnewline
2.4.1 & Amundsen Plains & 19528 & 26123 & 12571 & 5358 &
3879\tabularnewline
2.4.2 & Aberdeen Plains & 20284 & 7377 & 6876 & 5852 &
175\tabularnewline
2.4.3 & Central Ungava Peninsula and Ottawa and Belcher Islands & 12708
& 989 & 2019 & 2766 & 2328\tabularnewline
3.1.1 & Interior Forested Lowlands and Uplands & 12314 & 11610 & 12324 &
7956 & 1647\tabularnewline
3.1.2 & Interior Bottomlands & 8480 & 12243 & 12947 & 11485 &
3740\tabularnewline
3.3.2 & Hay and Slave River Lowlands & 18066 & 16530 & 18785 & 16477 &
2894\tabularnewline
3.4.2 & La Grande Hills and New Quebec Central Plateau & 23362 & 2721 &
289 & 7 & 687\tabularnewline
3.4.3 & Smallwood Uplands & 17695 & 3650 & 1900 & 1628 &
59\tabularnewline
3.4.5 & Coppermine River and Tazin Lake Uplands & 17076 & 31218 & 27986
& 32173 & 1788\tabularnewline
4.1.1 & Coastal Hudson Bay Lowland & 5247 & 6216 & 9476 & 4593 &
213\tabularnewline
4.1.2 & Hudson Bay and James Bay Lowlands & 19015 & 7396 & 1772 & 3408 &
5321\tabularnewline
5.1.1 & Athabasca Plain and Churchill River Upland & 17427 & 8784 & 9403
& 7999 & 5953\tabularnewline
5.1.2 & Lake Nipigon and Lac Seul Upland & 14113 & 7979 & 10656 & 10468
& 5219\tabularnewline
5.1.3 & Central Laurentians and Mecatina Plateau & 22320 & 22361 & 19161
& 19597 & 825\tabularnewline
5.1.4 & Newfoundland Island & 9737 & 7967 & 7865 & 7326 &
3273\tabularnewline
5.1.5 & Hayes River Upland and Big Trout Lake & 16227 & 10627 & 11089 &
13205 & 1022\tabularnewline
5.1.6 & Abitibi Plains and Riviere Rupert Plateau & 18020 & 9554 & 5260
& 5975 & 1\tabularnewline
5.2.1 & Northern Lakes and Forests & 19174 & 34982 & 36572 & 39567 &
42851\tabularnewline
5.2.2 & Northern Minnesota Wetlands & 2774 & 36 & 176 & 17 &
2\tabularnewline
5.2.3 & Algonquin/Southern Laurentians & 19758 & 19809 & 13374 & 14844 &
4684\tabularnewline
5.3.1 & Northern Appalachian and Atlantic Maritime Highlands & 13784 &
22199 & 21696 & 21793 & 14885\tabularnewline
5.4.1 & Mid-Boreal Uplands and Peace-Wabaska Lowlands & 26281 & 13199 &
7792 & 9977 & 2322\tabularnewline
5.4.2 & Clear Hills and Western Alberta Upland & 8944 & 987 & 610 & 824
& 114\tabularnewline
5.4.3 & Mid-Boreal Lowland and Interlake Plain & 8591 & 21433 & 22801 &
23930 & 17692\tabularnewline
6.1.1 & Interior Highlands and Klondike Plateau & 9208 & 3124 & 2530 &
1914 & 106\tabularnewline
6.1.2 & Alaska Range & 6968 & 3714 & 2627 & 2487 & 971\tabularnewline
6.1.4 & Wrangell and St.~Elias Mountains & 2554 & 1378 & 1049 & 831 &
10\tabularnewline
6.1.5 & Watson Highlands & 13655 & 10757 & 7553 & 6807 &
162\tabularnewline
6.1.6 & Yukon-Stikine Highlands/Boreal Mountains and Plateaus & 10586 &
5786 & 3741 & 3753 & 353\tabularnewline
6.2.1 & Skeena-Omineca-Central Canadian Rocky Mountains & 9171 & 8552 &
8820 & 9990 & 2291\tabularnewline
6.2.10 & Middle Rockies & 10794 & 6030 & 3226 & 2347 &
410\tabularnewline
6.2.11 & Klamath Mountains & 3253 & 3694 & 3332 & 3254 &
2232\tabularnewline
6.2.12 & Sierra Nevada & 3604 & 3118 & 3606 & 3832 & 2868\tabularnewline
6.2.13 & Wasatch and Uinta Mountains & 4243 & 813 & 238 & 623 &
63\tabularnewline
6.2.14 & Southern Rockies & 9092 & 7227 & 6959 & 6111 &
3052\tabularnewline
6.2.15 & Idaho Batholith & 3999 & 3728 & 3623 & 3928 &
2089\tabularnewline
6.2.2 & Chilcotin Ranges and Fraser Plateau & 6500 & 1795 & 1049 & 1413
& 84\tabularnewline
6.2.3 & Columbia Mountains/Northern Rockies & 11119 & 22325 & 24278 &
24061 & 22864\tabularnewline
6.2.4 & Canadian Rockies & 6511 & 1408 & 919 & 661 & 61\tabularnewline
6.2.5 & North Cascades & 2717 & 3194 & 3353 & 2192 & 1064\tabularnewline
6.2.6 & Cypress Upland & 837 & 556 & 282 & 395 & 885\tabularnewline
6.2.7 & Cascades & 2859 & 2817 & 3091 & 3011 & 3540\tabularnewline
6.2.8 & Eastern Cascades Slopes and Foothills & 3449 & 1616 & 861 & 741
& 200\tabularnewline
6.2.9 & Blue Mountains & 4304 & 6120 & 6600 & 6467 & 3588\tabularnewline
7.1.1 & Ahklun and Kilbuck Mountains & 3741 & 5592 & 9927 & 8813 &
9255\tabularnewline
7.1.2 & Alaska Peninsula Mountains & 4630 & 2073 & 1724 & 1409 &
1309\tabularnewline
7.1.3 & Cook Inlet & 2343 & 7889 & 6718 & 5350 & 2189\tabularnewline
7.1.4 & Pacific Coastal Mountains & 8158 & 5032 & 3887 & 3482 &
1811\tabularnewline
7.1.5 & Coastal Western Hemlock-Sitka Spruce Forests & 10176 & 13692 &
12519 & 12684 & 4999\tabularnewline
7.1.6 & Pacific and Nass Ranges & 7173 & 8014 & 7731 & 9747 &
3648\tabularnewline
7.1.7 & Strait of Georgia/Puget Lowland & 3380 & 3809 & 4980 & 5047 &
12675\tabularnewline
7.1.8 & Coast Range & 3878 & 3847 & 3489 & 3851 & 3272\tabularnewline
7.1.9 & Willamette Valley & 1129 & 2525 & 3088 & 2721 &
3538\tabularnewline
8.1.1 & Eastern Great Lakes Lowlands & 10611 & 27854 & 39218 & 39742 &
58844\tabularnewline
8.1.10 & Erie Drift Plain & 2274 & 307 & 62 & 6 & 54\tabularnewline
8.1.2 & Lake Erie Lowland & 4095 & 890 & 1202 & 1076 &
937\tabularnewline
8.1.3 & Northern Allegheny Plateau & 3060 & 36 & 2 & 4 &
130\tabularnewline
8.1.4 & North Central Hardwood Forests & 6153 & 7501 & 12214 & 11052 &
6399\tabularnewline
8.1.6 & Southern Michigan/Northern Indiana Drift Plains & 4974 & 1858 &
404 & 229 & 281\tabularnewline
8.1.7 & Northeastern Coastal Zone & 3483 & 16109 & 16031 & 17106 &
9656\tabularnewline
8.1.8 & Acadian Plains and Hills & 6882 & 1089 & 364 & 793 &
797\tabularnewline
8.1.9 & Maritime Lowlands & 3379 & 564 & 4 & 15 & 706\tabularnewline
8.2.1 & Southeastern Wisconsin Till Plains & 2639 & 1175 & 973 & 1252 &
6440\tabularnewline
8.2.2 & Huron/Erie Lake Plains & 3393 & 4880 & 5070 & 3660 &
4186\tabularnewline
8.2.3 & Central Corn Belt Plains & 5253 & 1871 & 1709 & 2247 &
4536\tabularnewline
8.2.4 & Eastern Corn Belt Plains & 5195 & 8044 & 6701 & 5080 &
614\tabularnewline
8.3.1 & Northern Piedmont & 2266 & 2042 & 2428 & 875 &
220\tabularnewline
8.3.2 & Interior River Valleys and Hills & 7204 & 18471 & 22834 & 32181
& 41053\tabularnewline
8.3.3 & Interior Plateau & 7863 & 4882 & 4497 & 3481 &
1847\tabularnewline
8.3.4 & Piedmont & 10363 & 2622 & 1337 & 895 & 173\tabularnewline
8.3.5 & Southeastern Plains & 15521 & 9623 & 12471 & 8427 &
3111\tabularnewline
8.3.6 & Mississippi Valley Loess Plains & 4937 & 811 & 522 & 1506 &
1092\tabularnewline
8.3.7 & South Central Plains & 9328 & 20114 & 18219 & 24925 &
36606\tabularnewline
8.3.8 & East Central Texas Plains & 3242 & 8719 & 9141 & 10183 &
11699\tabularnewline
8.4.1 & Ridge and Valley & 5920 & 904 & 578 & 203 & 422\tabularnewline
8.4.2 & Central Appalachians & 4867 & 832 & 448 & 138 &
119\tabularnewline
8.4.3 & Western Allegheny Plateau & 4778 & 947 & 617 & 703 &
13\tabularnewline
8.4.4 & Blue Ridge & 3021 & 1863 & 2013 & 2268 & 7292\tabularnewline
8.4.5 & Ozark Highlands & 6249 & 8596 & 4947 & 5703 &
5052\tabularnewline
8.4.6 & Boston Mountains & 1336 & 310 & 668 & 292 & 230\tabularnewline
8.4.7 & Arkansas Valley & 1969 & 3349 & 3055 & 2012 & 659\tabularnewline
8.4.8 & Ouachita Mountains & 1619 & 2407 & 1589 & 1016 &
710\tabularnewline
8.4.9 & Southwestern Appalachians & 3253 & 312 & 274 & 230 &
523\tabularnewline
8.5.1 & Middle Atlantic Coastal Plain & 7953 & 4024 & 7165 & 1763 &
5204\tabularnewline
8.5.2 & Mississippi Alluvial Plain & 8327 & 28746 & 33976 & 31813 &
33331\tabularnewline
8.5.3 & Southern Coastal Plain & 10023 & 14455 & 12420 & 11986 &
12766\tabularnewline
8.5.4 & Atlantic Coastal Pine Barrens & 1540 & 6493 & 9429 & 7146 &
10878\tabularnewline
9.2.1 & Aspen Parkland/Northern Glaciated Plains & 19219 & 43285 & 40756
& 47938 & 53002\tabularnewline
9.2.2 & Lake Manitoba and Lake Agassiz Plain & 7004 & 31809 & 43943 &
43028 & 71014\tabularnewline
9.2.3 & Western Corn Belt Plains & 13086 & 15829 & 14529 & 15256 &
36825\tabularnewline
9.2.4 & Central Irregular Plains & 7590 & 12209 & 7936 & 12002 &
2564\tabularnewline
9.3.1 & Northwestern Glaciated Plains & 22953 & 15444 & 13530 & 12853 &
23046\tabularnewline
9.3.3 & Northwestern Great Plains & 22220 & 26844 & 24463 & 22722 &
4034\tabularnewline
9.3.4 & Nebraska Sand Hills & 5161 & 281 & 272 & 251 & 40\tabularnewline
9.4.1 & High Plains & 17009 & 25323 & 27925 & 26333 &
13500\tabularnewline
9.4.2 & Central Great Plains & 14563 & 39122 & 52642 & 46814 &
122654\tabularnewline
9.4.3 & Southwestern Tablelands & 12148 & 19747 & 18592 & 20378 &
21736\tabularnewline
9.4.4 & Flint Hills & 2576 & 46 & 33 & 5 & 79\tabularnewline
9.4.5 & Cross Timbers & 5772 & 10147 & 11004 & 10631 &
26257\tabularnewline
9.4.6 & Edwards Plateau & 5712 & 155 & 44 & 39 & 14\tabularnewline
9.4.7 & Texas Blackland Prairies & 3141 & 1303 & 1032 & 1451 &
4601\tabularnewline
9.5.1 & Western Gulf Coastal Plain & 5505 & 14913 & 20089 & 20010 &
39523\tabularnewline
9.6.1 & Southern Texas Plains/Interior Plains and Hills with Xerophytic
Shrub and Oak Forest & 2942 & 13180 & 15724 & 17877 &
25204\tabularnewline
\bottomrule
\end{longtable}

I also specifically summarized changes for boreal ecoregions (5.4, 5.1,
3.4, 3.3, 3.2, 3.1, and 6.1):

Table 2. Model-projected changes by boreal ecoregion:

\begin{longtable}[]{@{}llrrrrr@{}}
\toprule
NA\_L3CODE & NA\_L3NAME & curr & rcp45\_2041.2070 & rcp45\_2071.2100 &
rcp85\_2041.2070 & rcp85\_2071.2100\tabularnewline
\midrule
\endhead
3.1.1 & Interior Forested Lowlands and Uplands & 12314 & 11610 & 12324 &
7956 & 1647\tabularnewline
3.1.2 & Interior Bottomlands & 8480 & 12243 & 12947 & 11485 &
3740\tabularnewline
3.3.2 & Hay and Slave River Lowlands & 18066 & 16530 & 18785 & 16477 &
2894\tabularnewline
3.4.2 & La Grande Hills and New Quebec Central Plateau & 23362 & 2721 &
289 & 7 & 687\tabularnewline
3.4.3 & Smallwood Uplands & 17695 & 3650 & 1900 & 1628 &
59\tabularnewline
3.4.5 & Coppermine River and Tazin Lake Uplands & 17076 & 31218 & 27986
& 32173 & 1788\tabularnewline
5.1.1 & Athabasca Plain and Churchill River Upland & 17427 & 8784 & 9403
& 7999 & 5953\tabularnewline
5.1.2 & Lake Nipigon and Lac Seul Upland & 14113 & 7979 & 10656 & 10468
& 5219\tabularnewline
5.1.3 & Central Laurentians and Mecatina Plateau & 22320 & 22361 & 19161
& 19597 & 825\tabularnewline
5.1.4 & Newfoundland Island & 9737 & 7967 & 7865 & 7326 &
3273\tabularnewline
5.1.5 & Hayes River Upland and Big Trout Lake & 16227 & 10627 & 11089 &
13205 & 1022\tabularnewline
5.1.6 & Abitibi Plains and Riviere Rupert Plateau & 18020 & 9554 & 5260
& 5975 & 1\tabularnewline
5.4.1 & Mid-Boreal Uplands and Peace-Wabaska Lowlands & 26281 & 13199 &
7792 & 9977 & 2322\tabularnewline
5.4.2 & Clear Hills and Western Alberta Upland & 8944 & 987 & 610 & 824
& 114\tabularnewline
5.4.3 & Mid-Boreal Lowland and Interlake Plain & 8591 & 21433 & 22801 &
23930 & 17692\tabularnewline
6.1.1 & Interior Highlands and Klondike Plateau & 9208 & 3124 & 2530 &
1914 & 106\tabularnewline
6.1.2 & Alaska Range & 6968 & 3714 & 2627 & 2487 & 971\tabularnewline
6.1.4 & Wrangell and St.~Elias Mountains & 2554 & 1378 & 1049 & 831 &
10\tabularnewline
6.1.5 & Watson Highlands & 13655 & 10757 & 7553 & 6807 &
162\tabularnewline
6.1.6 & Yukon-Stikine Highlands/Boreal Mountains and Plateaus & 10586 &
5786 & 3741 & 3753 & 353\tabularnewline
\bottomrule
\end{longtable}

This translates into 34\% and 83\% losses of boreal climate space by
2041-2070 and 2071-2100, respectively, based on RCP 8.5; or 27\% and
34\% losses based on RCP 4.5

\textbf{References}

Breiman, L. 2001. Random Forests. Machine Learning 45:5-32.

Commission for Environmental Cooperation. 1997. Ecological Regions of
North America: Toward a Common Perspective, Montreal, Canada.

Fuss, S., J. G. Canadell, G. P. Peters, M. Tavoni, R. M. Andrew, P.
Ciais, R. B. Jackson, C. D. Jones, F. Kraxner, N. Nakicenovic, C. Le
Quere, M. R. Raupach, A. Sharifi, P. Smith, and Y. Yamagata. 2014.
Betting on negative emissions. Nature Climate Change 4:850-853.

McKenney, D., J. Pedlar, M. Hutchinson, P. Papadopol, K. Lawrence, K.
Campbell, E. Milewska, R. F. Hopkinson, and D. Price. 2013. Spatial
climate models for Canada's forestry community. The Forestry Chronicle
89:659-663.

Rehfeldt, G. E., N. L. Crookston, C. Sáenz-Romero, and E. M. Campbell.
2012. North American vegetation model for land-use planning in a
changing climate: a solution to large classification problems.
Ecological Applications 22:119-141.

Taylor, K. E., R. J. Stouffer, and G. A. Meehl. 2012. An Overview of
CMIP5 and the Experiment Design. Bulletin of the American Meteorological
Society 93:485-498.


\end{document}
